\chapter{Fragen}
Aus den Auff\"alligkeiten ergeben sich folgende Fragen:
\begin{enumerate}
	\item Welche Bedeutung haben die 0 Werte in den Blickdateien?
	
	Bedeuten diese, dass die Versuchsperson die Augen geschlossen hatte?
	
	\textbf{Antwort:}\\
	\textit{0-Werte zeigen an, dass der Eyetracker kein Augensignal identifizieren konnte. Das kann durch geschlossene Augen (Blink) auftreten, es ist kann aber z.B. auch sein, dass die Distanz zwischen  Proband und Eyetracker zwischenzeitlich zu kurz/weit war.}
	
	\item Um welchen Wert sind die Koordinatensysteme verschoben?
	
	Gibt es eine Streckung des Koordinatensystems der Blickwerte zu dem der Targetwerte?
	
	\textbf{Antwort:}\\
	\textit{Die Bildschirmma\ss{}e waren 1280x1024 pixel. Das Koordinatensystem des Eyetrackers (Blickwerte) definiert (0,0) in der linken oberen Ecke. Das Koordinatensystem der Anzeigesoftware (Target) definiert (0,0) in der Mitte des Bildschirms und dementsprechend von -640...+640 entlang der xAchse und von -512...+512 entlang der y-Achse. Ansonsten sind die Koordinatensysteme identisch, d.h. keine relative Streckung zueinander.}
	
	\item K\"onnen wir davon ausgehen, dass die Zeitstempel der Blickdateien und der Targetdateien von synchron laufenden Uhren erstellt wurden?
	
	\textbf{Antwort:}\\
	\textit{Ja. Es ist in beiden F\"allen die interne Uhr des Eyetrackers.} 
	
	\item K\"onnen wir voraussetzen, dass der Fokus beim Sehen auf dem Mittelpunkt von der Blickposition des rechten und des linken Auges liegt?
	
	\textbf{Antwort:}\\
	\textit{Ja. Allerdings k\"onnte es durchaus sein, dass einige Probanden mit einem Auge mehr Ansteuerprobleme haben als mit dem anderen und die Augen daher nicht notwendigerweise synchron laufen.}
	
	
\end{enumerate}