\chapter{Auff\"alligkeiten}
\begin{itemize}
	\item Bei den Daten ist bei Versuchsperson 131 aufgefallen, dass in der Targetdatei die Eintr\"age\\
		\textit{MSG:UserQuit:ProgramPaused}\\
		\textit{MSG:UserQuit:ProgramRestarted}
		\\ in der Spalte f\"ur den Zeitstempel vorkommen.
	
	\item Die Blickpunkte wurden mit drei- bis vierfacher Frequenz der Targetpunkte gemssen. \\
		Dadurch gibt es zwischen den Targetpunktmessungen mehrere Blickpunktmessungen.
	
	\item Es gibt in den Blickpunktdateien keinen Zeitstempel, der identisch ist mit dem aus einer dazugeh\"origen Targetpunktedatei. Dadurch erschwert sich die Zuordnung der Blickpunkte zu den Targetpunkten.
	
	\item In den Blickpunktdateien tritt oft der Fall auf, dass die x-Werte und die y-Werte für die Augen 0 sind.
	
	\item Fehlende Werte bei den Blickdateien wurden mit 0 kodiert. Diese treten mal f\"ur das linke und mal f\"ur das rechte Auge auf.
	
	\item Bei der Visualisierung hat sich gezeigt, dass der Koordinatenursprung bei den Blickdateien am linken oberen Rand des Bildschirms zu sein scheint, mit aufsteigenden Werten nach unten auf der Y-Achse und aufsteigenden Werten nach rechts auf der X-Achse. Das Koordinatensystem der Targetdateien liegt vermutlich im Mittelpunkt des Bildschirms mit aufsteigenden Werten nach oben auf der Y-Achse und aufsteigenden Werten nach rechts auf der X-Achse. Das l\"asst sich daraus ableiten, dass die Blickpunkte bei der Animation auf der Y-Achse immer in die entgegengesetzte Richtung zu der der Targetpunkte gingen. Auf der X-Achse stimmte die Richtung immer überein.
	
	\item Die Blickdateien haben keine Headerzeile
	
	\item Die Zeitstempel sind nicht \"aquidistant
	
	\item Die Zeilenumbr\"uche in den Blickdateien, waren anders als die in den Targetdateien
	
	\item Blickpunkte wurden durchg\"angig gemessen, die Targetpunkte nur w\"ahrend der Versuche
	
	\item Einige Versuchspersonen haben nur Aufzeichnungen zu einem Auge (z.B. VP071)
	
	Da 3 Versuchspersonen besondere Auff\"alligkeiten vorzuweisen hatten, wurden diese nicht weiter verarbeitet. Das betrifft VP071 und VP090, da diese beiden nur mit einem Auge gemessen wurden, und VP131, da bei der Versuchsperson neu gestartet wurde.
	
\end{itemize}