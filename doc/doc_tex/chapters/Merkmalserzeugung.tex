\chapter{Merkmalserzeugung}
Bei der Merkmalserzeugung muss zwischen Merkmalen unterschieden werden, die innerhalb der Zeitreihen liegen und Merkmalen, die f\"ur die Gesamtbeschreibung der Versuchsperson genutzt werden.
\section{Abgeleitete Wertereihen}
Ein Merkmal innerhalb der Zeitreihen ist zum Beispiel die Mitte zwischen Blickposition des linken Auges und Blickposition des rechten Auges. Dabei wird f\"ur jeden Zeitpunkt im Datenstrom ein jeweiliger Wert berechnet.

Tabelle \ref{tab:MerkmaleZeitreihe} zeigt die Merkmale, die erzeugt werden k\"onnen.

\begin{table}[H]
	\caption{\label{tab:MerkmaleZeitreihe}Merkmale innerhalb der Zeitreihe}
	
	
	\noindent \centering{}
	\bgroup
	\def\arraystretch{2}  %  1 ist der Standardwert
	\begin{tabular}{|p{7.5cm}|p{7.5cm}|}
		\hline 
		\textbf{Merkmal} & \textbf{Berechnung}\\ \hline
		Mitte Augenpositionen & \begin{tabular}{c|c}
			$x=\frac{x_{links} + x_{rechts}}{2}$  & $y=\frac{y_{links} + y_{rechts}}{2}$ 
		\end{tabular} \\ \hline
		Abweichung Augenposition\newline linkes Auge zu Targetpunkt & $s_l=\sqrt{{\left(x_{links}-x_{target}\right)}^2+{\left(y_{links}-y_{target}\right)}^2}$ \\ \hline
		Abweichung Augenposition\newline rechtes Auge zu Targetpunkt & $s_r=\sqrt{{\left(x_{rechts}-x_{target}\right)}^2+{\left(y_{rechts}-y_{target}\right)}^2}$ \\ \hline
		Abweichung Mitte\newline Augenposition zu Targetpunkt & $s_m=\sqrt{{\left(x_{mitte}-x_{target}\right)}^2+{\left(y_{mitte}-y_{target}\right)}^2}$ \\ \hline
		Geschwindigkeit linkes Auge & $v_l=\frac{\sqrt{{\left(x_{links_1}-x_{links_2}\right)}^2+{\left(y_{links_1}-y_{links_2}\right)}^2}}{\left(zeitstempel_1-zeitestempel_2 \right) }$ \\ \hline
		Geschwindigkeit rechtes Auge & $v_r=\frac{\sqrt{{\left(x_{rechts_1}-x_{rechts_2}\right)}^2+{\left(y_{rechts_1}-y_{rechts_2}\right)}^2}}{\left(zeitstempel_1-zeitestempel_2 \right) }$ \\ \hline
		Geschwindigkeit Mittelposition Augen & $v_m=\frac{\sqrt{{\left(x_{mitte_1}-x_{mitte_2}\right)}^2+{\left(y_{mitte}-y_{mitte}\right)}^2}}{\left(zeitstempel_1-zeitestempel_2 \right) }$ \\ \hline
	\end{tabular}
	\egroup
\end{table}

Des Weiteren kann bestimmt werden, ob das Auge hinter dem Targetpunkt ist, oder davor. Dazu wird aus der Differenz eines Targetpunkts und seines Vorg\"angers bestimmt, ob sich der Punkt pro Achse in aufsteigende Richtung oder in absteigende Richtung bewegt. Wenn sich der Punkt beispielsweise in auf der x-Achse in Richtung aufsteigende Werte bewegt, dann bedeutet ein Blickpunkt mit einem gr\"o\ss{}eren x-Wert, dass der Blick vor dem Targetpunkt ist.
Wenn sich die Werte f\"ur einen Targetpunkt nicht \"andern, z.B. bei dem Versuch horizontal, auf der y-Achse, dann kann dieser Wert nicht bestimmt werden.

\section{Merkmale Versuchsperson}