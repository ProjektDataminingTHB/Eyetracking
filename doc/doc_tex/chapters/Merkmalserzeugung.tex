\chapter{Merkmalserzeugung}
Bei der Merkmalserzeugung muss zwischen Merkmalen unterschieden werden, die innerhalb der Zeitreihen liegen und Merkmalen, die f\"ur die Gesamtbeschreibung der Versuchsperson genutzt werden.
\section{Abgeleitete Wertereihen}
Ein Merkmal innerhalb der Zeitreihen ist zum Beispiel die Mitte zwischen Blickposition des linken Auges und Blickposition des rechten Auges. Dabei wird f\"ur jeden Zeitpunkt im Datenstrom ein jeweiliger Wert berechnet.

Tabelle \ref{tab:MerkmaleZeitreihe} zeigt die Merkmale, die erzeugt werden k\"onnen.

\begin{table}[H]
	\caption{\label{tab:MerkmaleZeitreihe}Merkmale innerhalb der Zeitreihe}
	
	\noindent \centering{}
	\bgroup
	\def\arraystretch{2}  %  1 ist der Standardwert
	\begin{tabular}{|p{7.5cm}|p{7.5cm}|}
		\hline 
		\textbf{Merkmal} & \textbf{Berechnung}\\ \hline
		Mitte Augenpositionen & \begin{tabular}{c|c}
			$x=\frac{x_{links} + x_{rechts}}{2}$  & $y=\frac{y_{links} + y_{rechts}}{2}$ 
		\end{tabular} \\ \hline
		Abweichung Augenposition\newline linkes Auge zu Targetpunkt & $s_l=\sqrt{{\left(x_{links}-x_{target}\right)}^2+{\left(y_{links}-y_{target}\right)}^2}$ \\ \hline
		Abweichung Augenposition\newline rechtes Auge zu Targetpunkt & $s_r=\sqrt{{\left(x_{rechts}-x_{target}\right)}^2+{\left(y_{rechts}-y_{target}\right)}^2}$ \\ \hline
		Abweichung Mitte\newline Augenposition zu Targetpunkt & $s_m=\sqrt{{\left(x_{mitte}-x_{target}\right)}^2+{\left(y_{mitte}-y_{target}\right)}^2}$ \\ \hline
		Geschwindigkeit linkes Auge & $v_l=\frac{\sqrt{{\left(x_{links_1}-x_{links_2}\right)}^2+{\left(y_{links\_1}-y_{links_2}\right)}^2}}{\left(zeitstempel_1-zeitestempel_2 \right) }$ \\ \hline
		Geschwindigkeit rechtes Auge & $v_r=\frac{\sqrt{{\left(x_{rechts_1}-x_{rechts_2}\right)}^2+{\left(y_{rechts_1}-y_{rechts_2}\right)}^2}}{\left(zeitstempel_1-zeitestempel_2 \right) }$ \\ \hline
		Geschwindigkeit Mittelposition Augen & $v_m=\frac{\sqrt{{\left(x_{mitte_1}-x_{mitte_2}\right)}^2+{\left(y_{mitte}-y_{mitte}\right)}^2}}{\left(zeitstempel_1-zeitestempel_2 \right) }$ \\ \hline
	\end{tabular}
	\egroup
\end{table}

Des Weiteren kann bestimmt werden, ob das Auge hinter dem Targetpunkt ist, oder davor. Dazu wird aus der Differenz eines Targetpunkts und seines Vorg\"angers bestimmt, ob sich der Punkt pro Achse in aufsteigende Richtung oder in absteigende Richtung bewegt. Wenn sich der Punkt beispielsweise in auf der x-Achse in Richtung aufsteigende Werte bewegt, dann bedeutet ein Blickpunkt mit einem gr\"o\ss{}eren x-Wert, dass der Blick vor dem Targetpunkt ist.
Wenn sich die Werte f\"ur einen Targetpunkt nicht \"andern, z.B. bei dem Versuch horizontal, auf der y-Achse, dann kann dieser Wert nicht bestimmt werden.

\section{Merkmale Versuchsperson}

Ein Merkmal zu einer Versuchsperson ist ein statistischer Wert \"uber den gesamten Zeitraum der Messung. Diese Werte sind aus den Zeitreihen abgeleitet. In der Regel handelt es sich um die Werte Maximum, Minimum, Durchschnitt, Median, Varianz, Standardabweichung.
Die genannten statistischen Kenngr\"o"sen werden
\begin{enumerate}
	\item f\"ur die Werteverteilung der Zeitreihen
	\item f\"ur die einzelnen Versuche
\end{enumerate}
erzeugt.

Au"serdem k\"onnen die Merkmale unterschieden werden in Merkmale, die ausschlie"slich aus den Blickdaten gewonnen werden. Diese enthalten dann auch Werte, die zwischen den einzelnen Versuchen entstanden sind. Und die Merkmale, die beim Vergleich zwischen Targetdaten und Blickdaten entstehen. Diese enthalten dann ausschlie"slich Daten, die w\"ahrend den Versuchen entstanden sind.

Die Tabelle \ref{tab:MerkmaleVP} benennt die Merkmale und gibt eine Beschreibung dazu. Die Tabelle enth\"alt nur die Merkmale f\"ur den Versuch Horizontal, da die weiteren Merkmale, die gleichen Merkmale sind, die allerdings f\"ur die anderen beiden Versuche erzeugt wurden.

\begin{longtable}{|p{7cm}|p{7.4cm}|}
	\hline
	Person &
	Bezeichnung der Versuchsperson\\ \hline
	Horizontal\_mean\_delta\_l &
	Arithmetischer Mittelwert der Abst\"ande des Blickpunkts vom linken Auge zum Targetpunkt im Versuch Horizontal\\ \hline
	Horizontal\_mean\_delta\_r &
	Arithmetischer Mittelwert der Abst\"ande des Blickpunkts vom rechten Auge zum Targetpunkt im Versuch Horizontal\\ \hline
	Horizontal\_mean\_delta\_m &
	Arithmetischer Mittelwert der Abst\"ande der Mittelung der Blickpunkte vom rechten und linken Auge zum Targetpunkt im Versuch Horizontal\\ \hline
	Horizontal\_mean\_geschwindigkeit\_l &
	Arithmetischer Mittelwert der Geschwindigkeit des linken Auges im Versuch Horizontal\\ \hline
	Horizontal\_mean\_geschwindigkeit\_r &
	Arithmetischer Mittelwert der Geschwindigkeit des rechten Auges im Versuch Horizontal\\ \hline
	Horizontal\_mean\_geschwindigkeit\_m &
	Arithmetischer Mittelwert der Geschwindigkeit der Mittelung der Augen im Versuch Horizontal\\ \hline
	Horizontal\_max\_delta\_l &
	Maximaler Abstand zwischen Blickposition des linken Auges und dem Targetpunkt im Versuch Horizontal\\ \hline
	Horizontal\_max\_delta\_r &
	Maximaler Abstand zwischen Blickposition des rechten Auges und dem Targetpunkt im Versuch Horizontal\\ \hline
	Horizontal\_max\_delta\_m &
	Maximaler Abstand zwischen Blickposition der Mittelung der Augenpositionen und dem Targetpunkt im Versuch Horizontal\\ \hline
	Horizontal\_max\_geschwindigkeit\_l &
	Maximale Geschwindigkeit des linken Auges im Versuch Horizontal\\ \hline
	Horizontal\_max\_geschwindigkeit\_r &
	Maximale Geschwindigkeit des rechten Auges im Versuch Horizontal\\ \hline
	Horizontal\_max\_geschwindigkeit\_m &
	Maximale Geschwindigkeit der Mittelung der Augen im Versuch Horizontal\\ \hline
	Horizontal\_min\_delta\_l &
	Minimaler Abstand zwischen Blickposition des linken Auges und dem Targetpunkt im Versuch Horizontal\\ \hline
	Horizontal\_min\_delta\_r &
	Minimaler Abstand zwischen Blickposition des rechten Auges und dem Targetpunkt im Versuch Horizontal\\ \hline
	Horizontal\_min\_delta\_m &
	Minimaler Abstand zwischen Blickposition der Mittelung der Augenpositionen und dem Targetpunkt im Versuch Horizontal\\ \hline
	Horizontal\_min\_geschwindigkeit\_l &
	Minimale Geschwindigkeit des linken Auges im Versuch Horizontal\\ \hline
	Horizontal\_min\_geschwindigkeit\_r &
	Minimale Geschwindigkeit des rechten Auges im Versuch Horizontal\\ \hline
	Horizontal\_min\_geschwindigkeit\_m &
	Minimale Geschwindigkeit der Mittelung der Augen im Versuch Horizontal\\ \hline
	Horizontal\_standardabweichung\_delta\_l &
	Standardabweichung der Abst\"ande des Blickpunkts vom linken Auge zum Targetpunkt im Versuch Horizontal\\ \hline
	Horizontal\_standardabweichung\_delta\_r &
	Standardabweichung der Abst\"ande des Blickpunkts vom rechten Auge zum Targetpunkt im Versuch Horizontal\\ \hline
	Horizontal\_standardabweichung\_delta\_m &
	Standardabweichung der Abst\"ande der Mittelung der Blickpunkte vom rechten und linken Auge zum Targetpunkt im Versuch Horizontal\\ \hline
	Horizontal\_standardabweichung\newline\_geschwindigkeit\_l &
	Standardabweichung der Geschwindigkeit des linken Auges im Versuch Horizontal\\ \hline
	Horizontal\_standardabweichung\newline\_geschwindigkeit\_r &
	Standardabweichung der Geschwindigkeit des rechten Auges im Versuch Horizontal\\ \hline
	Horizontal\_standardabweichung\newline\_geschwindigkeit\_m &
	Standardabweichung der Geschwindigkeit der Mittelung der Augen im Versuch Horizontal\\ \hline
	Horizontal\_varianz\_delta\_l &
	Varianz der Abst\"ande des Blickpunkts vom linken Auge zum Targetpunkt im Versuch Horizontal\\ \hline
	Horizontal\_varianz\_delta\_r &
	Varianz der Abst\"ande des Blickpunkts vom rechten Auge zum Targetpunkt im Versuch Horizontal\\ \hline
	Horizontal\_varianz\_delta\_m &
	Varianz der Abst\"ande der Mittelung der Blickpunkte vom rechten und linken Auge zum Targetpunkt im Versuch Horizontal\\ \hline
	Horizontal\_varianz\_geschwindigkeit\_l &
	Varianz der Abst\"ande des Blickpunkts vom linken Auge zum Targetpunkt im Versuch Horizontal\\ \hline
	Horizontal\_varianz\_geschwindigkeit\_r &
	Varianz der Abst\"ande des Blickpunkts vom rechten Auge zum Targetpunkt im Versuch Horizontal\\ \hline
	Horizontal\_varianz\_geschwindigkeit\_m &
	Varianz der Geschwindigkeit der Mittelung der Augen im Versuch Horizontal\\ \hline
	Horizontal\_tendenz\_l &
	Tendenz des linken Auges voraus (1), oder hinterher (-1) zu sein, im Versuch Horizontal\\ \hline
	Horizontal\_tendenz\_r &
	Tendenz des rechten Auges voraus (1), oder hinterher (-1) zu sein, im Versuch Horizontal\\ \hline
	Horizontal\_tendenz\_m &
	Tendenz der Mittelung der Augenpositionen voraus (1), oder hinterher(-1) zu sein, im Versuch Horizontal\\ \hline
	Horizontal\_Kovarianz\_blick\_x &
	Kovarianz der Blickpositionen beider Augen in x-Richtung im Versuch Horizontal\\ \hline
	Horizontal\_Kovarianz\_blick\_y &
	Kovarianz der Blickpositionen beider Augen in y-Richtung im Versuch Horizontal\\ \hline
	
	\caption{\label{tab:MerkmaleVP}Merkmale einer Versuchsperson}
	
\end{longtable}

