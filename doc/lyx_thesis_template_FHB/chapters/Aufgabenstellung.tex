\chapter{Aufgabenstellung}

In einem Experiment wurden Eyetracking-Daten erhoben, bei denen die Versuchspersonen drei verschiedene Versuche durchf\"uhren sollten. Dabei sollten die Versuchsperson mit den Augen einem Punkt folgen, der eine spezielle Figur zeichnete. Diese Figuren sind eine liegende acht und eine horizontale Linie.

\begin{itemize}
	\item Der erste Versuch ist die liegende acht langsam (acht Sekunden f\"ur einen Durchlauf).
	\item Der zweite Versuch ist die liegende acht schnell (vier Sekunden f\"ur einen Durchlauf).
	\item Der dritte Versuch ist die horizontale Linie (vier Sekunden f\"ur einen Durchlauf).
\end{itemize}

F\"ur jeden Versuch wurden zwei Messungen gemacht und f\"ur die liegende Acht langsam wurde zus\"atzlich ein Probedurchlauf gemacht.
Die Aufgabe besteht darin die Versuchspersonen zu gruppieren (clustern). Dabei sollen mit Hilfe der erhobenen Daten Merkmale gefunden werden, die es erm\"oglichen Gruppen zu bilden.
Zu dem Ergebnis geh\"oren folgende Bestandteile:

\begin{enumerate}
	\item Die Cluster
	\item Die Beschreibungen der Cluster
	\item Die Merkmale, die erzeugt wurden
\end{enumerate}

Die Aufgabe wird seit dem 30.11.2016 bearbeitet.