
\subsection{Ausgelagerter Unterabschnitt}

\LyX{} sorgt bei eingebundenen Dateien automatisch daf�r, dass die
richtige Nummerierung bei den �berschriften verwendet wird. Auch Quellenangaben
mit Quellen aus dem Hauptdokument sind leicht m�glich: \cite{Harel1987}.

Wird eine augelagerte Datei als \texttt{Input} eingebunden, wird der
Quelltext der Datei vor dem kompilieren in das Hauptdokument �bernommen.
Bei gro�en Dokumenten kann das zu Performance-Problemen f�hren. Aus
diesem Grund sollten ganze Kapitel besser als \texttt{Include} eingebunden
werden. Dabei wird erst das Ergebnis der Kompilierung zusammengef�hrt.
Der Nachteil ist, dass per \texttt{Include} eingebundene Dateien immer
auf einer neuen Seite beginnen und auch am Ende der eingebundenen
Datei eine neue Seite begonnen wird. Bei ganzen Kapiteln ist das jedoch
unerheblich, da diese sowieso auf einer neuen Seite beginnen.
