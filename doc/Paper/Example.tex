\documentclass[a4paper,twoside]{article}

\usepackage{epsfig}
\usepackage{subfigure}
\usepackage{calc}
\usepackage{amssymb}
\usepackage{amstext}
\usepackage{amsmath}
\usepackage{amsthm}
\usepackage{multicol}
\usepackage{pslatex}
\usepackage{apalike}
\usepackage{SciTePress}
\usepackage[small]{caption}

\subfigtopskip=0pt
\subfigcapskip=0pt
\subfigbottomskip=0pt

\begin{document}

\title{\uppercase{Authors' Instructions}  \subtitle{Preparation of Camera-Ready Contributions to SciTePress Proceedings} }

\author{\authorname{Mario Kaulmann\sup{1}, Herval Nganya\sup{1}}
\affiliation{\sup{1}University of Applied Sciences, Technische Hochschule Brandenburg, Magdeburger Stra\ss{}e 50, 14770 Brandenburg an der Havel, Deutschland}
}

\keywords{Data Mining, Clustern, Eye-Tracking}

\abstract{In dieser Arbeit werden Versuchspersonen mittels Eye-Trackingdaten geclustert. Diese Versuchspersonen sollten bei der Datenerfassung drei Veruche nacheinander durchf\"uhren. Bei diesen Versuchen sollte ein Punkt mit dem Blick verfolgt werden. Bei der Clusterung soll sich herausbilden, wie gut die Versuchspersonen diese Aufgabe gel\"ost haben. Die Bemessung der G\"ute der Cluster erfolgt mit Hilfe des Silhouettenkoeffizienten.}

\onecolumn \maketitle \normalsize \vfill

\section{\uppercase{Einf\"uhrung}}

\noindent 
In einem Experiment wurden Eyetracking-Daten erhoben, bei denen die Versuchspersonen drei verschiedene Versuche durchf\"uhren sollten. Dabei sollten die Versuchsperson mit den Augen einem Punkt folgen, der eine spezielle Figur zeichnete. Diese Figuren sind eine liegende Acht und eine horizontale Linie. F\"ur jeden Versuch wurden zwei Durchl\"aufe gemacht. Pro Durchlauf wurde die entsprechende Figur zwei mal gezeichnet. F\"ur die liegende Acht langsam wurde zus\"atzlich vorher ein Probedurchlauf gemacht, bei dem die Figur nur einmal gezeichnet wurde.
Die Tabelle \ref{tab:Versuche} zeigt die Versuche, die durchgef\"uhrt wurden.

\begin{table}[h]
	\caption{\label{tab:Versuche}Liste der Versuche: Hier wird die Reihenfolge der Versuche angegeben, sowie die Figur, die der Punkt ge\-zeich\-net hat und die Dauer eines Durchlaufs.}
	\noindent \centering{}
	\bgroup
	\def\arraystretch{2}  %  1 ist der Standardwert
	\begin{tabular}{|l|l|l|}
		\hline 
		\textbf{Reihenfolge} & \textbf{Figur} & \textbf{Dauer}\\
		\hline \hline
		1 & liegende Acht & 8 Sekunden\\
		\hline
		2 & liegende Acht & 4 Sekunden\\
		\hline
		3 & horizontale Linie & 4 Sekunden\\
		\hline
	\end{tabular}
	\egroup
\end{table}

Die erhobenen Eye-Trackingdaten sind Zeitreihen. Zu jeder Versuchsperson gibt es eine Datei mit den Blickpunktdaten der Person und eine Datei mit den Koordinaten des Punkts, der verfolgt werden sollte. Mittels dieser Zeitreihen werden Merkmale erzeugt, die zur Clusterung der Versuchspersonen genutzt werden k\"onnen. Die Cluster sollen dabei widerspiegeln, wie gut die Versuchspersonen die Aufgabe gel\"ost haben.

\section{\uppercase{Datenbeschreibung}}
\noindent
Insgesamt liegen Daten von 302 Versuchspersonen vor. Zu jeder Versuchsperson gibt es eine Datei mit Blickdaten (Blickdatei) und eine Datei mit den Zielpunktdaten (Zieldatei). Die Tabelle \ref{tab:AttrBlickdatei} zeigt die Attribute einer Blickdatei. Diese enth\"alt auch Werte, die ignoriert werden k\"onnen.

\begin{table}[h]
	\caption{\label{tab:AttrBlickdatei}Attribute Blickdatei: Der Attributwert ist eine Beschreibung, dessen was das Attribut ausdr\"uckt. Der Wert gibt den Datentyp an, oder dass das Attribut f\"ur diese Aufgabe ignoriert werden kann.}
	\noindent \centering{}
	\bgroup
	\def\arraystretch{2}  %  1 ist der Standardwert
	\begin{tabular}{|l|l|}
		\hline 
		\textbf{Attribut} & \textbf{Wert}\\
		\hline \hline
		Zeitstempel & Ganze Zahl positiv --> Zeitreihen\\
		\hline
		Blick linkes Auge (x) & Flie\ss{}kommazahl \\
		\hline
		Blick linkes Auge (y) & Flie\ss{}kommazahl \\
		\hline
		Pupillengr\"o\ss{}e linkes Auge & Kann ignoriert werden \\
		\hline
		Pos. linkes Auge vor Eyetracker (x) & Kann ignoriert werden \\
		\hline
		Pos. linkes Auge vor Eyetracker (y) & Kann ignoriert werden \\
		\hline
		Dist. linkes Auge vor Eyetracker & Kann ignoriert werden \\
		\hline
		Blick rechtes Auge (x) & Flie\ss{}kommazahl \\
		\hline
		Blick rechtes Auge (y) & Flie\ss{}kommazahl \\
		\hline
		Pupillengr\"o\ss{}e rechtes Auge & Kann ignoriert werden \\
		\hline
		Pos. rechtes Auge vor Eyetracker (x) & Kann ignoriert werden \\
		\hline
		Pos. rechtes Auge vor Eyetracker (y) & Kann ignoriert werden \\
		\hline
		Dist. rechtes Auge vor Eyetracker & Kann ignoriert werden \\
		\hline
	\end{tabular}
	\egroup
\end{table}

\section{\uppercase{Vorgehensweise}}
\noindent 

\section{\uppercase{Ergebnisse}}
\noindent 

\section{\uppercase{Zusammenfassung}}
\noindent 

\section*{\uppercase{Danksagung}}

\vfill
\bibliographystyle{apalike}
{\small
\bibliography{example}}

\end{document}

